\documentclass[a4paper]{article}

\usepackage[spanish]{babel}
\usepackage[utf8]{inputenc}
\usepackage{amsmath}
\usepackage{graphicx}
%\usepackage[colorinlistoftodos]{todonotes}
\usepackage{hyperref}
\usepackage{multicol}
\usepackage{makeidx}
\usepackage{hyperref}
\usepackage{caption}
\usepackage{amsfonts}
\usepackage{amssymb}
\usepackage[utf8]{inputenc}
\usepackage{verbatim}
\usepackage{listings}
\usepackage{float}
\lstset{language=C++, showstringspaces=false, tabsize=2, breaklines=true, title=\lstname}

\usepackage[margin=0.5in]{geometry}

\title{TP1 Tleng}

\author{-}

\date{\today}

\makeindex

\begin{document}
\newgeometry{margin=2cm}
\pagenumbering{gobble}
\raggedleft
\includegraphics[width=8cm]{logo_dc.jpg}\\

\raggedright
\vspace{3cm}
{\Huge \bfseries Trabajo Práctico 2}
\rule{\textwidth}{0.02in}
\large Miércoles 29 de julio de 2015 \hfill Teoría de Lenguajes
\vspace{1.5cm}

 
\centering \LARGE 
\vspace{1.5cm}

\normalsize
\begin{tabular}{|l@{\hspace{4ex}}c@{\hspace{4ex}}l|}
        \hline
        \rule{0pt}{1.2em}Integrante & LU & Correo electr\'onico\\[0.2em]
        \hline
        \rule{0pt}{1.2em} Aleman, Damián Eliel &377/10 &\tt damian\_8591@hotmail.com \\[0.2em]
        \rule{0pt}{1.2em} Gauna, Claudio Andrés &733/06 &\tt gauna\_claudio@yahoo.com.ar\\[0.2em]        
        \hline
\end{tabular}

\vspace{1.0cm}
\raggedright

\begin{multicols}{2}
\includegraphics[width=8cm]{logo_uba.jpg}

\columnbreak
\vspace*{4.5cm}
\raggedleft
\textbf{Facultad de Ciencias Exactas y Naturales}\\
Universidad de Buenos Aires\\
\small
Ciudad Universitaria - (Pabellon I/Planta Baja)\\
Intendente G\"uiraldes 2160 - C1428EGA\\
Ciudad Autonoma de Buenos Aires - Rep. Argentina\\
Tel/Fax: (54 11) 4576-3359\\
http://www.fcen.uba.ar
\end{multicols}

\restoregeometry

\clearpage

\pagenumbering{arabic}

\tableofcontents

\vspace{3cm}

\clearpage


\section{Introducción}
El objetivo del trabajo practico es implementar un parser para un lenguaje orientado a la composicion de piezas musicales, llamado Musileng, que luego sera transformado al formato MIDI 1 para su reproduccion.

Los pasos que seguimos para realizar el trabajo practico fueron:

\begin{itemize}
\item Generar la gramatica adecuada, que sintetice el lenguaje orientado a la composicion de las piezas musicales
\item Escribir los terminales del lenguaje y las reglas del lexer
\item Escribir los no termianles del lengueje y las reglas del parser.
\item Agregarle semantica para que pueda imprimir al lenguaje intermedio que pueda ser leido por el programa midcomp
\item Luego de finalizada la traduccion, pueda transformarse a MIDI (.mid) por medio del programa midicomp
\end{itemize}

Utilizamos para el trabajo la herramienta ANTLR para generar el parser (y el lexer) necesitado.


\section{Grámatica}

A continuación definimos la gramática que generamos para sintetizar el lenguaje:
Las terminales de la gramática son todas las cadenas que están entre comillas simples y los que se derivan a partir de una regla ( de los tokens en mayúscula).



S : tempos elcompas constantes melodia \\
tempos : '\#tempo' DURACION NUM \\
elcompas :  '\#compas' NUM '/' NUM \\
constantes :  constante* \\
constante :  const texto = NUM \\
melodia :  ('voz''('NUM')' \{ compases\} )+ $|$ ('voz''('texto')' \{ compases\} )+ \\
compases : compas compases $|$ repeticion compases \\
repeticion : 'repetir''('NUM')' '{'compas'}' \\
compas : ('compas' '{'nota'}')+ $|$ ('compas' '('silencio')')+ \\
silencio : 'silencio''('DURACION PUNTILLO?')' \\
nota : 'nota' '('ALTURA ALTERACION?, octava, DURACION PUNTILLO?')' \\
octava : OCTAVA $|$ TEXTO \\
ALTERACION : '+' $|$'-' ; \\
PUNTILLO : '.'; \\
DURACION : ('blanca'$|$'negra'$|$'corchea'$|$'semicorchea'$|$'fusa'$|$'semifusa'); \\
ALTURA : ('do'$|$'re'$|$'mi'$|$'fa'$|$'sol'$|$'la'$|$'si'); \\
NUM : [0-9]+; \\
TEXTO : $[a-zA-Z_]$+; \\
OCTAVA : [1-9] $|$ '1'[0-5]; \linebreak



El símbolo distinguido es S.\linebreak


El conjunto de no terminales es:\linebreak
\texttt{
\{S, tempos, elcompas,contantes, contante, melodia, compases, repeticion, compas, silencio,nota, octava\}  \\
} 

\section{Atributos}

Asignamos atributos para verificar las restricciones que tenemos que hacer de modo tal que la gramatica genere el lenguaje que necesitamos.

El conjunto de terminales es: \\
\texttt{
\{\#tempo/, \#compas, =, ;, ., -, +, (, ), \{, \}, ,,  const, voz, repetir, nota, silencio,  alteracion, puntillo, duracion, Altura, Num, TExto, OCTAVA\}
}\linebreak

Los atributos sintetizados son:\\
\texttt{
\{partitura, tempo, indicacion, listaCompases, voces, repeticiones, compasObj, silencioObj,notaObj, valor\}
}\\  


Los atributos heredados son: \\
\texttt{
\{indicacion\}
}\\  

Ahora haremos una breve explicacion de cada atributo:

partitura:



\end{document}
