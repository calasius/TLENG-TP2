\documentclass[a4paper]{article}

\usepackage[spanish]{babel}
\usepackage[utf8]{inputenc}
\usepackage{amsmath}
\usepackage{graphicx}
%\usepackage[colorinlistoftodos]{todonotes}
\usepackage{hyperref}
\usepackage{multicol}
\usepackage{makeidx}
\usepackage{hyperref}
\usepackage{caption}
\usepackage{amsfonts}
\usepackage{amssymb}
\usepackage[utf8]{inputenc}
\usepackage{verbatim}
\usepackage{listings}
\usepackage{float}
\lstset{language=C++, showstringspaces=false, tabsize=2, breaklines=true, title=\lstname}

\usepackage[margin=0.5in]{geometry}

\title{TP1 Tleng}

\author{-}

\date{\today}

\makeindex

\begin{document}
\newgeometry{margin=2cm}
\pagenumbering{gobble}
\raggedleft
\includegraphics[width=8cm]{logo_dc.jpg}\\

\raggedright
\vspace{3cm}
{\Huge \bfseries Trabajo Práctico 2}
\rule{\textwidth}{0.02in}
\large Miércoles 29 de julio de 2015 \hfill Teoría de Lenguajes
\vspace{1.5cm}

 
\centering \LARGE 
\vspace{1.5cm}

\normalsize
\begin{tabular}{|l@{\hspace{4ex}}c@{\hspace{4ex}}l|}
        \hline
        \rule{0pt}{1.2em}Integrante & LU & Correo electr\'onico\\[0.2em]
        \hline
        \rule{0pt}{1.2em} Aleman, Damián Eliel &377/10 &\tt damian\_8591@hotmail.com \\[0.2em]
        \rule{0pt}{1.2em} Gauna, Claudio Andrés &733/06 &\tt gauna\_claudio@yahoo.com.ar\\[0.2em]        
        \hline
\end{tabular}

\vspace{1.0cm}
\raggedright

\begin{multicols}{2}
\includegraphics[width=8cm]{logo_uba.jpg}

\columnbreak
\vspace*{4.5cm}
\raggedleft
\textbf{Facultad de Ciencias Exactas y Naturales}\\
Universidad de Buenos Aires\\
\small
Ciudad Universitaria - (Pabellon I/Planta Baja)\\
Intendente G\"uiraldes 2160 - C1428EGA\\
Ciudad Autonoma de Buenos Aires - Rep. Argentina\\
Tel/Fax: (54 11) 4576-3359\\
http://www.fcen.uba.ar
\end{multicols}

\restoregeometry

\clearpage

\pagenumbering{arabic}

\tableofcontents

\vspace{3cm}

\clearpage


\section{Introducción}
El objetivo del trabajo practico es implementar un parser para un lenguaje orientado a la composicion de piezas musicales, llamado Musileng, que luego sera transformado al formato MIDI 1 para su reproduccion.

Los pasos que seguimos para realizar el trabajo practico fueron:

\begin{itemize}
\item Generar la gramatica adecuada, que sintetice el lenguaje orientado a la composicion de las piezas musicales
\item Escribir los terminales del lenguaje y las reglas del lexer
\item Escribir los no termianles del lengueje y las reglas del parser.
\item Agregarle semantica para que pueda imprimir al lenguaje intermedio que pueda ser leido por el programa midcomp
\item Luego de finalizada la traduccion, pueda transformarse a MIDI (.mid) por medio del programa midicomp
\end{itemize}

Utilizamos para el trabajo la herramienta ANTLR para generar el parser (y el lexer) necesitado.


\section{Grámatica}

A continuación definimos la gramática que generamos para sintetizar el lenguaje:
Las terminales de la gramática son todas las cadenas que están entre comillas simples y los que se derivan a partir de una regla ( de los tokens en mayúscula).



S $\rightarrow$ tempos elcompas constantes melodia \\
tempos $\rightarrow$ '\#tempo' DURACION NUM \\
elcompas $\rightarrow$  '\#compas' NUM '/' NUM \\
constantes $\rightarrow$  constante* \\
constante $\rightarrow$  const texto = NUM \\
melodia $\rightarrow$  ('voz''('NUM')' \{ compases\} )+ $|$ ('voz''('texto')' \{ compases\} )+ \\
compases $\rightarrow$ compas compases $|$ repeticion compases \\
repeticion $\rightarrow$ 'repetir''('NUM')' '{'compas'}' \\
compas $\rightarrow$ ('compas' '{'nota'}')+ $|$ ('compas' '('silencio')')+ \\
silencio $\rightarrow$ 'silencio''('DURACION PUNTILLO?')' \\
nota $\rightarrow$ 'nota' '('ALTURA ALTERACION?, octava, DURACION PUNTILLO?')' \\
octava $\rightarrow$ OCTAVA $|$ TEXTO \\
ALTERACION $\rightarrow$ '+' $|$'-' ; \\
PUNTILLO $\rightarrow$ '.'; \\
DURACION $\rightarrow$ ('blanca'$|$'negra'$|$'corchea'$|$'semicorchea'$|$'fusa'$|$'semifusa'); \\
ALTURA $\rightarrow$ ('do'$|$'re'$|$'mi'$|$'fa'$|$'sol'$|$'la'$|$'si'); \\
NUM $\rightarrow$ [0-9]+; \\
TEXTO $\rightarrow$ $[a-zA-Z\_]$+; \\
OCTAVA $\rightarrow$ [1-9] $|$ '1'[0-5]; \linebreak



El símbolo distinguido es S.\linebreak


El conjunto de no terminales es:\linebreak
\texttt{
\{S, tempos, elcompas,contantes, contante, melodia, compases, repeticion, compas, silencio,nota, octava\}  \\
} 

\section{Atributos}

Asignamos atributos para verificar las restricciones que tenemos que hacer de modo tal que la gramatica genere el lenguaje que necesitamos.
Los atributos tambien los usaremos para realizar la traduccion al lenguaje intermedio para que sea legible por el programa midcomp.

El conjunto de terminales es: \\
\texttt{
\{\#tempo/, \#compas, =, ;, ., -, +, (, ), \{, \}, ,,  const, voz, repetir, nota, silencio,  alteracion, puntillo, duracion, Altura, Num, TExto, OCTAVA\}
}\linebreak

Los atributos sintetizados son:\\
\texttt{
\{partitura, tempo, indicacion, listaCompases, voces, repeticiones, compasObj, silencioObj,notaObj, valor\}
}\\  


Los atributos heredados son: 
\texttt{
\{indicacion\}
}\\  

Ahora haremos una breve explicacion de cada atributo:

\begin{itemize}
\item partitura: almacena el tempo, la indicacion y la lista de voces. 

\item tempo: Este atributo guarda la informacion de la duracion de la figura y la cantidad de veces que entra esa figura en un minuto.

\item indicacion: Almacena el numerador y el denominador definidos del compas.

\item voces:La lista de voces de la melodia

\item listaCompases: la lista de los compasObj y las repetcionesObj

\item compasObj: Tiene la lista de notas  de cada compas
\item repeticiones: almacena la cantidad de repeticiones
\item notaObj: Almacena la altura,de que octava es, la duracion, la alteracion y si tiene puntillo
\item silencioObj: Tiene la lista de silencios de cada compas
\item valor: el valor de la octava 
\item listacompases: Tiene todos los compases definidos

\end{itemize}





\end{document}
